\section{System Architecture and Workflow}

The implemented system follows a modular architecture that connects frontend, backend, database, and external language model services into a coherent workflow. At its core, the system realises the four-agent pipeline described in Chapter~3, while ensuring that each agent is technically orchestrated through the backend. Communication between components is carried out using \textbf{JSON} messages, which serve as a transparent and extensible exchange format.

\subsection*{Workflow}
The overall workflow proceeds as follows:

\begin{enumerate}
    \item A user provides input through the \textbf{frontend interface}, either by uploading an audio file or initiating a live recording.
    \item The audio is transmitted to the \textbf{backend}, which invokes the \textbf{Whisper API} for transcription. The result is a JSON object containing the raw transcript and metadata.
    \item The transcript is passed to the \textbf{Information Extraction Agent}, which calls either OpenAI’s ChatGPT or Google’s Gemini. Depending on the chosen strategy, this step may involve a single LLM call, multiple slot-wise calls, or parallel consensus queries. The output is a structured JSON object with proposed field values.
    \item The \textbf{Consistency Formatting Agent} applies standardisation rules, ensuring uniform date formats, canonicalised entity names, and harmonised terminology.
    \item The \textbf{Verification Agent} validates the results by checking for contradictions, completeness, and plausibility. In case of low confidence or detected inconsistencies, clarification requests are triggered.
    \item The verified template is stored in the \textbf{PostgreSQL database}, managed via Sequelize, and made accessible through the backend.
    \item Finally, the structured template is visualised in the \textbf{React frontend}, where users can review, correct, or export results.
\end{enumerate}

\subsection*{Architecture Diagram}
Figure~\ref{fig:system-architecture} shows the system architecture, with a vertical layout optimised for readability on A4 pages.


\begin{figure}[H]
\centering
\resizebox{0.8\linewidth}{!}{%
\begin{tikzpicture}[
  every node/.style={font=\sffamily},
  box/.style={draw, rounded corners=2pt, thick, minimum width=56mm, minimum height=12mm,
              align=center, fill=blue!7},
  ext/.style={draw, rounded corners=2pt, thick, minimum width=52mm, minimum height=10mm,
              align=center, fill=gray!10},
  group/.style={draw, rounded corners=3pt, thick, inner sep=6pt, fill=white},
  arrow/.style={-Latex, thick},
  dsh/.style={-Latex, thick, dashed},
  lab/.style={font=\scriptsize, inner sep=1pt},
  node distance=12mm
]

% MAIN VERTICAL SPINE
\node[box] (fe) {Frontend (React + shadcn/ui)};
\node[box, below=of fe] (be) {Backend Orchestrator (Node.js + Express + tRPC)};
\node[box, below=of be] (db) {PostgreSQL + Sequelize};
\node[group, below=of db, label={[lab]above:{Four-Agent Pipeline (JSON-based)}}] (agents) {};
  % agents inner boxes (stacked)
  \node[box, minimum width=50mm, anchor=north] (stt) at ([yshift=-6pt]agents.north) {STT};
  \node[box, minimum width=50mm, below=6mm of stt] (ie)  {Information Extraction (IE)};
  \node[box, minimum width=50mm, below=6mm of ie] (cf)  {Consistency Formatting (CF)};
  \node[box, minimum width=50mm, below=6mm of cf] (ver) {Verification (VER)};
  \node[fit=(stt)(ie)(cf)(ver), draw=none] (agentsfit) {};

\node[box, below=12mm of ver] (out) {Verified Structured Template (JSON)};

% EXTERNAL SERVICES (sides)
\node[ext, right=24mm of stt] (whisper) {Whisper API (ASR)};
\node[ext, right=24mm of ie]  (llms) {LLM Providers (ChatGPT, Gemini)};
\node[ext, right=24mm of be]  (auth) {Authentication (Keycloak)};

% FLOWS (90-degree connectors only)
\draw[arrow] (fe) -- node[lab,right]{tRPC + JSON} (be);
\draw[arrow] (be) -- node[lab,right]{JSON} (db);
\draw[arrow] (db) -- node[lab,right]{JSON} (agentsfit.north);

\draw[arrow] (stt) -- (ie);
\draw[arrow] (ie) -- (cf);
\draw[arrow] (cf) -- (ver);
\draw[arrow] (ver) -- (out);

% External integrations (right-angle only)
\draw[arrow] (be.east) -| (auth.west);
\draw[arrow] (stt.east) -| (whisper.west);
\draw[arrow] (ie.east)  -| (llms.west);
\draw[arrow] (ver.east) -| (llms.west);

% DB persistence of final output (right-angle)
\draw[arrow] (out.west) -| ([xshift=-20mm]db.west) |- (db.west);

% Clarification loop (dashed, right-angles)
\draw[dsh] (ver.west) -| ([xshift=-20mm]ie.west) |- (ie.west)
            node[lab,pos=0.25,left]{Clarify (Text)};
\draw[dsh] (ver.west) -| ([xshift=-20mm]stt.west) |- (stt.west)
            node[lab,pos=0.25,left]{Clarify (Speech)};

\end{tikzpicture}%
}
\caption{Implementation architecture: vertical, JSON/tRPC spine with four agents. External services (Keycloak, Whisper, LLMs) connect via right-angle links. Verification can trigger clarification to IE (text) or STT (speech). The final JSON template is persisted to PostgreSQL and returned to the frontend.}
\label{fig:impl-architecture}
\end{figure}


\subsection*{Discussion}
The architecture demonstrates how the conceptual agents are embedded within a practical software system. The vertical flow clarifies the progression from input to verified output, while JSON ensures transparent communication across all layers. The modular separation also enables independent refinement of components: for example, alternative LLMs can be substituted without altering the frontend or database. This separation of concerns provides both flexibility for experimentation and robustness for deployment.
