\section{Technology Stack and Rationale}

The implementation of the \textit{Invox} system required a careful analysis and choice of technologies that could support the modular multi-agent pipeline while ensuring scalability, reliability, and ease of development. The final stack combines modern web frameworks, robust database solutions, authentication infrastructure, and state-of-the-art language models. This section provides a detailed description of the selected components and the rationale behind each choice.

\subsection*{Backend}
The backend of the system is implemented in \textbf{Node.js} using the \textbf{Express} framework. Node.js was selected for its non-blocking I/O model and its strong ecosystem for building scalable web services. Express provides a lightweight but extensible framework for defining routes and middleware, making it suitable for orchestrating the different agents of the pipeline.  

To achieve type-safe and efficient communication between the frontend and backend, the system employs \textbf{tRPC}. Unlike traditional REST APIs, which require separate schema definitions and validation, tRPC enables end-to-end type safety by sharing type definitions between client and server. This reduces boilerplate code, minimizes runtime errors, and supports rapid iteration during development. All inter-agent messages and client–server interactions are exchanged in \textbf{JSON} format to maintain a consistent and human-readable structure.

\subsection*{Frontend}
The user interface is developed in \textbf{React}, a widely adopted library for building modular, component-based web applications. React’s virtual DOM and state management features make it suitable for rendering dynamic content such as structured templates and extracted fields. The system integrates the \textbf{shadcn/ui} library to provide a set of modern, accessible, and customizable UI components. This reduces frontend development effort while ensuring a consistent design language across the application.

\subsection*{Database}
For persistent storage, the system relies on \textbf{PostgreSQL}, a relational database well-suited for handling structured data with complex relationships. The slot-filling task in MUC-4 maps naturally to relational structures, where fields such as incident type, date, location, and perpetrator can be represented as structured columns or as \texttt{JSONB} objects when more flexibility is required. To facilitate schema definition and object mapping, the system uses the \textbf{Sequelize} ORM. Sequelize simplifies database migrations, ensures secure query generation, and abstracts away low-level SQL operations, which accelerates development without sacrificing control.

\subsection*{Authentication}
Authentication and role-based access control are managed through \textbf{Keycloak}, an open-source identity and access management system. Keycloak supports OAuth2 and OpenID Connect, enabling secure integration with the backend and frontend. It provides essential features such as single sign-on, token-based authentication, and fine-grained role assignment. This is particularly relevant in collaborative scenarios where different roles (e.g., system administrators, annotators, or evaluators) require distinct levels of access to system functionality.

\subsection*{Language Models and AI Services}
The system integrates multiple large language model services to support different stages of the pipeline. \textbf{OpenAI Whisper} is used for automatic speech recognition, ensuring that spoken inputs are reliably converted into textual transcripts. For information extraction, both \textbf{ChatGPT (OpenAI)} and \textbf{Gemini (Google)} are employed. ChatGPT provides high-quality natural language understanding, while Gemini introduces diversity and enables ensemble or consensus-based strategies. The integration of multiple models reflects the modular philosophy of the system, allowing experimentation with single-model and multi-model approaches as outlined in the conceptual strategies.

\subsection*{Development Environment and Tooling}
Development was carried out using \textbf{Visual Studio Code} as the primary integrated development environment, chosen for its extensibility and strong ecosystem of plugins for JavaScript, TypeScript, and Node.js development. \textbf{Postman} was employed for testing and debugging API endpoints during backend development. Version control and project organisation followed a \textbf{monorepo} structure under Git, where backend, frontend, and database components are maintained as submodules. This organisation simplifies dependency management, enforces coherence across components, and supports streamlined deployment.

\subsection*{Rationale}
The chosen stack reflects a balance between robustness, modularity, and experimental flexibility. Node.js with Express provides a reliable foundation for the backend, React enables dynamic and modular frontend design, and PostgreSQL ensures robust handling of structured template data. Keycloak secures the system through modern authentication standards, while Whisper, ChatGPT, and Gemini provide state-of-the-art natural language processing capabilities. The integration of these components allows the conceptual architecture to be realised in a practical, extensible system that supports experimentation with different extraction and verification strategies.
