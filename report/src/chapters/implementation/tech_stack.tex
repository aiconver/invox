\section{Technology Stack and Rationale}
\label{sec:tech-stack}

The implementation of the \textit{Invox} system employs a modular monolith architecture that balances development simplicity with operational flexibility. The core application executes as a single Node.js service containing all five agents, while specialized components such as vector search and embedding generation operate as external services. This hybrid approach preserves the benefits of modular design while avoiding the operational complexity typically associated with fully distributed microservice deployments \cite{fowler2015monolith,microservices_newman}.

\subsection*{Architecture Pattern: Modular Monolith with External Services}

The system follows a modular monolith pattern in which all agents (STT, RAG, IE, CF, VER) are implemented as cohesive modules within a single Node.js application. This design choice offers several advantages: a unified codebase that reduces coordination overhead, simplified debugging, and consistent versioning across agents. In-process inter-agent communication minimizes latency compared to network-based microservices, while modularity is maintained through well-defined interfaces that preserve separation of concerns.

External specialized services are integrated via network calls. \textbf{OpenSearch} provides vector similarity search, accessible through \texttt{http://opensearch:9200}; a dedicated embedding service generates dense vector representations; and external LLM APIs (OpenAI, Google) handle model inference. This hybrid approach leverages specialized infrastructure where necessary, while retaining the core template-filling logic within a unified application \cite{opensearch_knn}.

\subsection*{Containerization and Deployment}

The entire system is containerized using \textbf{Docker}. The main application runs in a single container, ensuring a consistent runtime environment across development and production. Containerization simplifies dependency management, enables horizontal scaling through replication, and facilitates seamless integration with external services. \textbf{Docker Compose} orchestrates the multi-service deployment, coordinating the main application container alongside OpenSearch and other dependencies \cite{docker_docs}.

\subsection*{Backend Implementation}

The backend is implemented in \textbf{Node.js} with \textbf{Express}, providing a robust foundation for the agent pipeline. The system employs the \textbf{Vercel AI SDK} to realize a plugin-like architecture for LLM integration. This abstraction enables seamless switching between AI providers (OpenAI, Google, Anthropic), simplifies the addition of new models with minimal code changes, and future-proofs the system against the rapid evolution of the AI ecosystem \cite{llm_orchestration_survey}. Type-safe communication between frontend and backend is achieved through \textbf{tRPC}, while \textbf{JSON} serves as a flexible exchange format for inter-agent communication.

\subsection*{External Service Integration}

\textbf{OpenSearch} functions as the vector database for the RAG agent, enabling efficient semantic search over historical templates. A dedicated embedding generation service produces dense vector representations, which underpin similarity matching. Both services expose \textbf{REST APIs}, maintaining clear service boundaries while enabling specialized functionality \cite{opensearch_knn}.

\subsection*{Frontend Interface}

The user interface is implemented in \textbf{React}, selected for its component-based architecture that aligns naturally with the modular design of the template-filling system. React's virtual DOM and state management capabilities enable efficient rendering of dynamic content, such as structured templates and extracted fields with confidence indicators.

To ensure modern and accessible UI design, the system integrates the \textbf{shadcn/ui} component library. This provides reusable, consistent components with built-in design principles, thereby reducing frontend development overhead while ensuring a professional user experience for template review and correction workflows.

\subsection*{Data Persistence and Retrieval}

Persistent storage is provided by \textbf{PostgreSQL}, a relational database well-suited for structured data with complex relationships. The MUC-4 template schema maps naturally onto relational structures: fields such as incident type, date, and location are represented as structured columns, while more flexible attributes are stored in \texttt{JSONB} to allow schema evolution \cite{postgresql_jsonb}.

The \textbf{Sequelize} ORM is employed for schema definition, object mapping, secure query generation, and migration management. For vector similarity search, \textbf{OpenSearch} with the \texttt{knn} plugin enables efficient retrieval of relevant historical templates using dense embeddings \cite{opensearch_knn}.

\subsection*{Authentication and Security}

Authentication and role-based access control are managed via \textbf{Keycloak}, an open-source identity and access management solution supporting OAuth2 and OpenID Connect \cite{rfc6749,openid_connect}. Keycloak ensures secure integration with backend services and allows fine-grained role assignments—essential for collaborative workflows in which different users require distinct levels of access to system functionality.

\subsection*{Architecture Rationale}

The modular monolith pattern was chosen as a pragmatic balance between simplicity and flexibility. It combines the development velocity of a monolith with the architectural clarity of modular design \cite{fowler2015monolith}. The plugin-oriented LLM integration enabled by the Vercel AI SDK ensures adaptability to new models and capabilities as the field evolves, while containerized deployment guarantees reliable operation at scale. This architecture thereby provides a future-proof, maintainable, and efficient foundation for the \textit{Invox} system.
