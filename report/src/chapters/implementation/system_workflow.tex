\section{System Workflow}
\label{sec:system-workflow}

This section describes the operational workflow of the Invox system, tracing the transformation of unstructured input into verified structured templates. The workflow implements the conceptual pipeline from Chapter~\ref{chap:concept} through a sequence of processing stages with user-driven quality control.

\subsection*{End-to-End Processing Flow}

The system workflow begins with user input and proceeds through six sequential stages:

\begin{enumerate}
    \item \textbf{Input Reception}: Users provide input through speech (audio recording) or direct text entry via the web interface. The system accepts both modalities and routes them appropriately through the processing pipeline.
    
    \item \textbf{Speech Recognition}: For audio inputs, the system invokes the Whisper ASR service to generate transcripts with confidence scores and timestamps. Text inputs proceed directly to the next stage.
    
    \item \textbf{Context Retrieval}: The RAG agent queries the vector database to retrieve the most relevant historical templates and examples, providing contextual guidance for the extraction process.
    
    \item \textbf{Information Extraction}: Based on the selected strategy (S1–S4), the system processes the input:
    \begin{itemize}
        \item \textbf{S1 (Single LLM)}: A single language model extracts all template fields in one pass
        \item \textbf{S2 (Slot-wise)}: Multiple LLM calls extract each field independently  
        \item \textbf{S3/S4 (Multi-LLM)}: Multiple models generate candidate answers, with a dedicated judge LLM selecting the optimal response through comparative analysis
    \end{itemize}
    
    \item \textbf{Consistency Enforcement}: The CF agent normalizes dates to ISO format, standardizes entity names, and enforces schema compliance through deterministic transformation rules.
    
    \item \textbf{Quality Assessment and Presentation}: The system assigns confidence scores to each extracted field and presents the complete template to the user for review and potential correction.
\end{enumerate}

\subsection*{User-Driven Quality Control}

The system employs a user-in-the-loop approach where quality assurance is primarily driven by human judgment:

\textbf{Confidence-Based Presentation}: Fields with lower confidence scores are visually highlighted, directing user attention to potentially problematic extractions.

\textbf{Multi-Model Decision Making}: In strategies S3 and S4, the judge LLM evaluates candidate answers from multiple models, selecting the most appropriate response based on coherence, accuracy, and alignment with the input context.

\textbf{User-Initiated Reprocessing}: If users are unsatisfied with results, they can modify their input and resubmit for reprocessing, creating an iterative refinement cycle driven by human assessment rather than automated detection.

\subsection*{Knowledge Accumulation}

Successfully processed templates are indexed in the vector database, enabling the RAG agent to leverage an expanding knowledge base for improved context retrieval in future processing cycles.