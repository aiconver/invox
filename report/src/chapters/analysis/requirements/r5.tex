Learning and adaptation refer to the framework’s ability to improve its performance over time by incorporating knowledge from user interactions, corrections, and structured domain information. Unlike static systems, which produce the same outputs regardless of experience, an adaptive framework evolves continuously. It refines its internal models and decision-making strategies based on provided examples, explicit corrections, and formal domain rules, thereby becoming more accurate and contextually aligned with the environment in which it is deployed~\cite{doshi2017towards, sun2023slot}.  

As illustrated in Figure~\ref{fig:form-filling-example}, finalized corrections and domain glossaries feed a feedback loop that informs future fills. This matters because using both sources reduces recurring errors and steadily improves accuracy.

This requirement matters because template filling systems are often deployed in highly specialized settings where general-purpose models cannot fully capture local terminology, workflows, or conventions. For example, a healthcare provider may require specific mappings between colloquial expressions such as “chest cold” and formal diagnostic categories, while a manufacturing plant may use internal identifiers or shorthand that differ from industry-wide terminology. Without learning and adaptation, the system risks repeating the same mistakes indefinitely, requiring users to make the same corrections again and again. By contrast, a system that consolidates knowledge from past interactions and domain inputs progressively reduces error rates and builds long-term efficiency~\cite{liu2022conversational, mialon2023augmented}.  

In practice, adaptation can occur through two complementary mechanisms. First, learning from provided filled-in templates allows the system to build expectations about the type and structure of values that belong in each field. For example, if historical forms consistently show that an employee ID consists of exactly five characters, the system can detect when a user provides only three and prompt them to complete the entry correctly. Over time, such feedback reduces recurring formatting or completeness errors by aligning new inputs with established patterns. Second, learning from domain knowledge integrates structured resources such as ontologies, glossaries, or formal rule sets. For instance, a medical ontology may specify that “flu” and “influenza” are synonymous terms, while an engineering glossary may define acceptable formats for torque values. By encoding such knowledge, the system improves both extraction and formatting, aligning outputs with established standards rather than ad-hoc interpretations~\cite{clark2020accessible, chen2024webforms}.

The advantages of learning and adaptation are cumulative. Accuracy improves progressively as the system incorporates examples and corrections, ensuring that the framework becomes more reliable the longer it is used in a given domain. Knowledge consolidation ensures that corrections made once benefit future cases, reducing the burden of repetitive manual adjustments. Continuous refinement also lowers the risk of systematic errors, as patterns of mistakes are gradually eliminated. These capabilities transform template filling from a static tool into an evolving system that adapts to organizational needs and grows more effective with sustained use~\cite{shneiderman2016designing, amershi2019guidelines}.  

\begin{table}[h!]
\centering
\renewcommand{\arraystretch}{1.6}
\setlength{\tabcolsep}{12pt}
\begin{tabularx}{\textwidth}{|>{\centering\arraybackslash}m{3cm}|>{\arraybackslash}X|}
\hline
\textbf{Visual Score} & \textbf{Interpretation} \\
\hline
\centering\raisebox{0pt}{\tikz[baseline]{\filldraw[fill=black] (0,0) circle (0.4cm);}} 
& \textbf{Both mechanisms satisfied.} The system learns from corrected or manually filled templates and incorporates structured domain knowledge such as ontologies and glossaries. \\
\hline
\centering\raisebox{0pt}{\tikz[baseline]{\filldraw[fill=black] (0,0) -- (90:0.4cm) arc (90:-90:0.4cm) -- cycle; \draw (0,0) circle (0.4cm);}} 
& One mechanism satisfied (e.g., the system learns from corrections but cannot integrate domain knowledge, or vice versa). \\
\hline
\centering\raisebox{0pt}{\tikz[baseline]{\draw (0,0) circle (0.4cm);}} 
& No mechanisms satisfied. The system is static, with no ability to learn from past usage or structured knowledge sources. \\
\hline
\end{tabularx}
\caption{Evaluation Scale for R5: Learning and Adaptation}
\label{tab:r5-learning}
\end{table}

To summarize, learning and adaptation ensure that the framework evolves with use, incorporating both user-provided corrections and structured domain knowledge. This capability reduces repetitive errors, consolidates expertise, and progressively improves accuracy, enabling the system to remain aligned with the specialized needs of its deployment context.
