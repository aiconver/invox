Consistency refers to the framework's ability to transform heterogeneous and unstructured inputs—such as transcribed voice recordings, instant messages between

employees, or informal written notes—into uniformly populated templates with standardized styling and formatting. 

Unlike free-text logs, templates require values that maintain consistent wording patterns, register, length conventions, and structural formatting. This is a core aspect of what the research field calls "authorship attribution" in the context of LLMs, where the goal is to identify a consistent stylistic "fingerprint" from a given system~\cite{huang2024authorship}. Prior work in conversational systems has shown that a lack of normalization makes it difficult to align user inputs across contexts~\cite{liu2022conversational}, while studies in accessibility highlight how inconsistent representations reduce clarity and comparability for end users~\cite{clark2020accessible}.

As illustrated in Figure~\ref{fig:form-filling-example}, an ideal solution should ensure that once information is extracted, the resulting field values are expressed in a stylistically consistent way. This means that answers follow standardized wording patterns, tone, formatting conventions, and structural organization. For example, dates should always appear in the same format, technical issues should be described with consistent terminology and register, and descriptive fields should maintain comparable stylistic patterns.

This requirement is especially critical because heterogeneous sources of input inevitably produce stylistically inconsistent outputs if left unchecked. In industrial contexts, for example, one technician might describe a machine malfunction as a "slight issue," another as a "minor fault," and a third as a "mechanical malfunction." Even if these phrases convey similar information, their stylistic inconsistency complicates automated reporting and trend analysis~\cite{norman2013design}. In healthcare, freeform reports such as "bad cough," "patient has a cold," or "respiratory illness" may differ in register and formality. Medical informatics research has emphasized that without stylistic normalization, such variations undermine clinical decision support~\cite{friedman2004survey, jonnalagadda2010medical}.

In practice, consistency operates across multiple dimensions of stylistic representation. At the level of wording, normalization ensures that common phrases follow standardized terminology patterns. At the level of tonality, consistency harmonizes register and formality, replacing subjective expressions such as "slight issue" with standardized formulations such as "minor mechanical malfunction." At the level of length, entries maintain comparable levels of detail. At the level of formatting, structural conventions ensure that dates, identifiers, and technical codes follow uniform patterns—for instance, "August 16th" is consistently represented as "08/16/2025."

The advantages of consistent template filling are multifold. Consistency improves readability, as users can process and compare entries more quickly when values follow uniform stylistic patterns. It reduces ambiguity by replacing stylistically divergent expressions with standardized equivalents. It enhances comparability: homogeneous field values are easier to query, index, and aggregate, facilitating automated reporting and large-scale data analysis.

For evaluation, we use exact match or F1 scores for structured fields (such as identifiers, categories, and dates) and n-gram-based stylometric analysis for free-text fields to assess stylistic alignment. Other similar benchmarks measuring formatting consistency and stylistic similarity can also be used depending on the domain and available resources. To evaluate the stylistic consistency of the system's output, we employ Cosine Similarity of n-gram distributions between each system's output and the golden benchmark. N-grams, which are contiguous sequences of words, provide a quantitative measure of the system's syntactic and stylistic patterns. Unlike Jaccard similarity, which only considers the presence or absence of n-grams, Cosine Similarity accounts for their frequency, making it sensitive to differences in writing style, register, and structural patterns. A higher similarity score indicates that the system's output is stylistically aligned with the human-authored standard.

\begin{table}[h!]
\centering
\renewcommand{\arraystretch}{1.6}
\setlength{\tabcolsep}{12pt}
\begin{tabularx}{\textwidth}{|>{\centering\arraybackslash}m{3cm}|>{\arraybackslash}X|}
\hline
\textbf{Visual Score} & \textbf{Interpretation (with example thresholds)} \\
\hline
\centering\raisebox{0pt}{\tikz[baseline]{\filldraw[fill=black] (0,0) circle (0.4cm);}} 
& \textbf{High consistency.} Structured fields follow formatting standards in most cases (e.g., macro exact match/F1 $\geq 0.80$), and free-text fields show strong stylistic alignment (e.g., Cosine Similarity $\geq 0.70$). \\
\hline
\centering\raisebox{0pt}{\tikz[baseline]{\filldraw[fill=black] (0,0) -- (90:0.4cm) arc (90:-150:0.4cm) -- cycle; \draw (0,0) circle (0.4cm);}} 
& \textbf{Medium consistency.} Structured fields show moderate formatting accuracy (e.g., macro exact match/F1 in $[0.65, 0.80)$), and free-text fields achieve moderate stylistic similarity (e.g., Cosine Similarity in $[0.55, 0.70)$). \\
\hline
\centering\raisebox{0pt}{\tikz[baseline]{\filldraw[fill=black] (0,0) -- (90:0.4cm) arc (90:-30:0.4cm) -- cycle; \draw (0,0) circle (0.4cm);}} 
& \textbf{Low consistency.} Structured fields often deviate from formatting standards (e.g., macro exact match/F1 in $[0.50, 0.65)$), and free-text fields show weak stylistic alignment (e.g., Cosine Similarity in $[0.40, 0.55)$). \\
\hline
\centering\raisebox{0pt}{\tikz[baseline]{\draw (0,0) circle (0.4cm);}} 
& \textbf{No consistency.} Structured formatting is very inconsistent (e.g., macro exact match/F1 $< 0.50$) and free-text stylistic similarity is weak (e.g., Cosine Similarity $< 0.40$). \\
\hline
\end{tabularx}
\caption{Evaluation scale for R1: Consistency (example thresholds).}
\label{tab:r1-consistency-thresholds}
\end{table}

To summarize, consistency ensures that heterogeneous, informal, or stylistically varied user inputs are transformed into standardized, uniformly formatted template entries. By enforcing normalization across wording patterns, tonality, length conventions, and formatting standards, and by measuring stylistic alignment through n-gram-based similarity metrics, the system improves readability and ensures that structured data maintains uniform presentation standards.