Information extraction refers to the framework’s ability to accurately identify and populate relevant template fields from unstructured or semi-structured input, while leaving irrelevant fields empty. This process goes beyond simple keyword matching and instead relies on linguistic context, entity relationships, and domain-specific semantics to ensure that extracted values are both accurate and meaningful~\cite{grishman1997information, sarawagi2008information}. In practice, information extraction determines whether heterogeneous and noisy inputs—such as partially transcribed voice recordings, multilingual text segments, or fragmented chat messages—can be transformed into reliable structured data~\cite{jurafsky2023speech, liu2022conversational}.  

As illustrated in Figure~\ref{fig:form-filling-example}, an ideal solution must be able to extract relevant details even when input is noisy, fragmented, or multilingual, and populate the corresponding template fields while leaving others empty. The highlighted gaps in the example show that extraction is not only about finding correct values but also about reliably detecting when information is missing and avoiding unsupported guesses.

This requirement matters because real-world text is rarely clean or complete. Voice transcriptions may drop words or introduce errors, employees in international environments may switch between languages mid-sentence, and workplace chat logs often contain fragments, abbreviations, or extraneous commentary~\cite{baldwin2013noisy}. Without strong extraction capabilities, systems risk either leaving critical fields blank or, worse, filling them with incorrect assumptions. In regulated or safety-critical domains such as healthcare or industrial maintenance, such errors can result in compliance violations, misdiagnoses, or operational risks~\cite{radford2023whisper, fathullah2023prompting}.  

Effective extraction must address several challenges simultaneously. First, it must be robust to mixed-language and incomplete input, requiring cross-lingual transfer capabilities for reliable extraction~\cite{ruder2019cross}. A report that states “der Patient… fever since yesterday” combines German and English, yet the system must still detect the symptom and correctly populate the appropriate field. Equally important is recognizing when essential details are missing and preserving empty fields instead of filling them with fabricated values. For example, if a date of inspection is not provided, the system should explicitly leave the field empty rather than guessing. Second, conflict handling is essential. If a transcript contains contradictory information—such as “appointment on August 10” followed by “meeting scheduled for August 12”—the system should not arbitrarily choose one date. Instead, it must detect the inconsistency and prompt the user for clarification, ensuring reliability of the final record. Third, extraction requires contextual understanding. Information is often scattered across multiple parts of a conversation or document, and the system must combine relevant fragments into coherent field values while ignoring unrelated chatter. For instance, if a technician states, “We restarted the press at 23:14. Oh, and the noise got louder after ten minutes,” the system must connect these details to the same inspection event rather than treating them as separate or unrelated entries.  

\begin{table}[h!]
\centering
\renewcommand{\arraystretch}{1.6}
\setlength{\tabcolsep}{12pt}
\begin{tabularx}{\textwidth}{|>{\centering\arraybackslash}m{3cm}|>{\arraybackslash}X|}
\hline
\textbf{Visual Score} & \textbf{Interpretation (based on criteria)} \\
\hline
\centering\raisebox{0pt}{\tikz[baseline]{\filldraw[fill=black] (0,0) circle (0.4cm);}} 
& \textbf{All three criteria satisfied.} The system reliably handles mixed-language and incomplete text, correctly detects and flags conflicts, and successfully integrates related information spread across different text parts. \\
\hline
\centering\raisebox{0pt}{\tikz[baseline]{\filldraw[fill=black] (0,0) -- (90:0.4cm) arc (90:-150:0.4cm) -- cycle; \draw (0,0) circle (0.4cm);}} 
& Two criteria satisfied. For instance, the system handles mixed-language input and detects conflicts but fails to integrate scattered contextual information. \\
\hline
\centering\raisebox{0pt}{\tikz[baseline]{\filldraw[fill=black] (0,0) -- (90:0.4cm) arc (90:-30:0.4cm) -- cycle; \draw (0,0) circle (0.4cm);}} 
& One criterion satisfied. For instance, the system handles mixed-language and incomplete text but fails to detect conflicts and integrate contextual information. \\
\hline
\centering\raisebox{0pt}{\tikz[baseline]{\draw (0,0) circle (0.4cm);}} 
& No criteria satisfied. The system fails to handle mixed-language input, cannot detect conflicts, and does not integrate information spread across different parts of the input. \\
\hline
\end{tabularx}
\caption{Evaluation Scale for R2: Information Extraction Abilities.}
\label{tab:r2-extraction-metrics}
\end{table}


The advantages of strong information extraction capabilities extend across multiple dimensions. They improve accuracy by minimizing both incorrect and missing field values, even when input is fragmented or linguistically complex. They reduce user workload by lowering the amount of manual correction required after automatic field population. They enhance robustness to noise by maintaining performance in the presence of typos, filler words, or irrelevant side remarks. Finally, they enable better multi-source integration by combining relevant information from different segments into a single coherent template entry, which is particularly valuable in collaborative or multi-turn interactions~\cite{shneiderman2016designing}.  

To summarize, information extraction abilities determine whether heterogeneous and noisy user inputs can be reliably transformed into structured templates. A robust system not only identifies and assigns relevant details but also recognizes gaps, resolves contradictions, and integrates scattered fragments into coherent entries. This reduces user workload, improves reliability, and ensures that the extracted data supports both operational decision-making and automated analysis.
