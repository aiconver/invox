Across the reviewed literature, each line of work contributes an essential piece to the broader challenge of structured information extraction. End-to-end generative models advance accuracy and reduce pipeline fragility; speech-focused methods bring robustness under noisy ASR conditions; schema-guided prompting stabilizes output formats; automated prompt optimization enhances reliability without training; and multi-agent systems introduce modularity, traceability, and division of labour. Taken together, these approaches reveal a clear trajectory: from brittle rule-based systems to flexible but opaque single-model LLMs, and finally toward coordinated, role-specialized multi-agent architectures.

Yet, despite this progress, the consolidated evaluation shows that no existing approach satisfies all six requirements (R1--R6). Transparency, user-centered correction, adaptive learning, and measurable usability remain underdeveloped across the field, while many methods excel only in narrow subsets of requirements. This gap highlights the absence of an integrated solution capable of handling noisy, speech-driven inputs while remaining interpretable, correctable, and operationally robust.

These limitations directly motivate the system design developed in this thesis. Chapter~3 therefore builds on the insights from this review to derive a modular architecture that combines the strengths of prior work—generative extraction, schema enforcement, iterative refinement, and multi-agent coordination—while explicitly addressing the unmet requirements identified here. The following chapter translates this analysis into concrete design principles, architectural decisions, and processing pipelines that underpin the proposed Invox system.
