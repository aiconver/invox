Across the reviewed literature, each line of work contributes an essential piece to the broader challenge of structured information extraction. End-to-end generative models improve accuracy and reduce pipeline fragility; speech-focused methods add robustness under noisy ASR conditions; schema-guided prompting stabilizes formats; automated prompt optimization increases reliability; and multi-agent systems introduce modularity, traceability, and division of labour. Together, these developments trace a trajectory from brittle rule-based systems to flexible but opaque single-model LLMs, and finally toward coordinated, role-specialized multi-agent architectures.

Despite this progress, the consolidated evaluation shows that no existing approach satisfies all six requirements (R1--R6). Transparency, user-centered correction, adaptive learning, and measurable usability remain underdeveloped, and most methods excel only in narrow subsets of requirements. This reveals a clear gap: the field lacks an integrated solution that handles noisy, speech-driven inputs while remaining interpretable, correctable, and operationally robust.

These limitations directly motivate the system design developed in this thesis. Chapter~3 builds on these insights to propose a modular architecture that combines the strengths of prior work—generative extraction, schema enforcement, iterative refinement, and multi-agent coordination—while explicitly addressing the unmet requirements identified here. The following chapter translates this analysis into concrete design principles, architectural decisions, and processing pipelines behind the proposed Invox system.
