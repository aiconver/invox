\section{System Overview}
\label{sec:concept-overview}

The \textit{Invox} system converts unstructured natural language into structured template fields through a modular, four–agent architecture. Audio input is supported via a speech interface, but the core contribution is the reliable mapping from unstructured text to well-defined template slots. The design is intentionally componentised to avoid the brittleness and opacity of single, monolithic LLM pipelines.

\subsection{Goal and Scope}
This chapter presents the concept of the solution---its decomposition, data flow, and orchestration patterns. The target task is filling the fields of a predefined template (e.g., the MUC-4 schema) from narrative text. Evaluation methodology and metrics are not discussed here and are deferred to the later Evaluation chapter.

\subsection{Four-Agent Pipeline}
The pipeline is organised as four specialised agents:
\begin{itemize}
    \item \textbf{Speech-to-Text (STT).} Optional front-end for audio input. A Whisper-based adapter produces text for downstream components. As the thesis focuses on text-to-template mapping, this stage remains deliberately brief.
    \item \textbf{Information Extraction (IE).} The central component. It analyses the text, identifies relevant entities and events, and proposes values for each template slot.
    \item \textbf{Consistency Formatting (CF).} It normalises the proposed values into canonical forms (e.g., date formats, location names, label vocabularies) and enforces simple schema rules.
    \item \textbf{Verification (VER).} It checks completeness and cross-field consistency, highlights conflicts or low-confidence fields, and may request clarifications.
\end{itemize}

\subsection{Processing Flow}
Inputs are refined stage by stage. Audio follows \textit{STT} $\rightarrow$ \textit{IE} $\rightarrow$ \textit{CF} $\rightarrow$ \textit{VER}. Text bypasses STT and enters at IE. Each step reduces ambiguity and increases structure, producing machine-actionable output that conforms to the target schema. When VER detects gaps or contradictions, the system can trigger a clarification round with the user and then re-apply CF and VER on the updated values.

\subsection{Execution Modes}
While the pipeline order is fixed for correctness, the IE and VER stages admit two execution modes that motivate the architectural strategies developed later:
\begin{enumerate}
    \item \textbf{Sequential (single-pass).} One inference computes the full set of fields, followed by CF and VER.
    \item \textbf{Slot-parallel (iterative).} Fields are processed independently, allowing parallel extraction per slot and subsequent aggregation before CF and VER. This reduces latency on multi-field templates and isolates errors.
\end{enumerate}
Both modes are compatible with single-LLM and multi-LLM deployments. The concrete strategy (Section~\ref{sec:concept-strategies}) determines whether one model handles all fields, fields are processed one by one, or multiple models are combined by a consensus mechanism.

\subsection{Design Advantages}
The modular design yields three practical benefits:
\begin{itemize}
    \item \textbf{Transparency.} Intermediate artefacts at each stage (candidate spans, normalised values, verification notes) make decisions traceable and simplify error analysis.
    \item \textbf{Isolated improvements.} Components can be tuned or replaced without destabilising the whole system (e.g., stricter normalisation or a stronger verifier).
    \item \textbf{Flexible deployment.} The same architecture supports different LLM deployment patterns (single-pass, iterative, or consensus) to balance quality, cost, and latency.
\end{itemize}

\subsection{Structure of the Remainder}
Section~\ref{sec:concept-architecture} details the four agents and their interfaces. Section~\ref{sec:concept-strategies} specifies four architectural strategies (single-pass full input, iterative single-field, multi-LLM consensus over full input, and multi-LLM consensus per field) using the same pipeline. Implementation details follow in the next chapter.



\begin{figure}[H]
\centering
\resizebox{\linewidth}{!}{%
\scriptsize
\begin{tikzpicture}[
  node distance=8mm,
  ext/.style={draw,rounded corners,align=center,inner sep=2pt,minimum width=24mm,minimum height=7mm,fill=gray!10},
  sys/.style={draw,rounded corners=3pt,thick,align=center,inner sep=3pt,minimum width=36mm,minimum height=9mm,fill=blue!7},
  rel/.style={-Latex,thick},
  note/.style={font=\scriptsize\itshape,align=center}
]
\node[ext] (user) {User};
\node[sys, right=24mm of user] (invox) {Invox System};
\node[ext, above right=4mm and 18mm of invox] (asr) {ASR Provider\\(Whisper)};
\node[ext, right=22mm of invox] (llm) {LLM Providers};
\node[ext, below right=4mm and 18mm of invox] (idp) {Identity Provider\\(OIDC)};

\draw[rel] (user) -- node[above, note]{Speech / Text} (invox);
\draw[rel] (invox) -- node[above, note]{Transcribe} (asr);
\draw[rel] (invox) -- node[above, note]{Inference} (llm);
\draw[rel, dashed] (invox) -- node[right, note]{Auth / Tokens} (idp);
\draw[rel, dashed] (user) |- node[pos=0.25,left, note]{Review / Corrections} (invox);

\node[fit=(user)(invox)(asr)(llm)(idp), draw, rounded corners, inner sep=4pt] {};
\end{tikzpicture}%
}
\caption{C4 context: Invox and its external actors/services (no datasets or evaluation).}
\label{fig:c4-context}
\end{figure}



\begin{figure}[H]
\centering
\resizebox{\linewidth}{!}{%
\scriptsize
\begin{tikzpicture}[
  node distance=7mm and 9mm,
  box/.style={draw,rounded corners,thick,align=left,inner sep=3pt,minimum width=38mm,minimum height=9mm,fill=blue!7},
  ext/.style={draw,rounded corners,align=center,inner sep=2pt,minimum width=30mm,minimum height=7mm,fill=gray!10},
  sub/.style={draw,rounded corners=2pt,align=left,inner sep=2pt,minimum width=34mm,fill=white},
  rel/.style={-Latex,thick},
  dashedrel/.style={-Latex,thick,dashed},
  group/.style={draw,rounded corners,inner sep=4pt}
]
% External
\node[ext] (client) {Client UI (Web/App)};
\node[ext, right=74mm of client] (llms) {LLM APIs};
\node[ext, below=14mm of llms] (asr) {ASR API (Whisper)};
\node[ext, above=14mm of llms] (idp) {Identity Provider (OIDC)};

% Backend group
\node[group, right=16mm of client, minimum width=74mm, minimum height=52mm, label={[align=left]north:Invox Backend}] (grp) {};
\node[box, anchor=north west] (gw) at ([xshift=4mm,yshift=-4mm]grp.north west) {API Gateway / Auth};
\node[box, below=5mm of gw] (orch) {Orchestrator};
\node[sub, below=2mm of orch] (agents) {Agents:\\
-- Speech-to-Text Adapter\\
-- Information Extraction\\
-- Consistency Formatting\\
-- Verification};
\node[sub, right=8mm of orch] (store) {Internal Storage:\\
-- Prompt / Result Cache\\
-- Template Schema (slots \& rules)\\
-- Logs / Traces};

% Edges
\draw[rel] (client) -- node[above]{HTTPS} (gw);
\draw[rel] (gw) -- (orch);
\draw[rel] (orch) -- (agents);
\draw[rel] (orch) -- node[above]{KV / DB} (store);
\draw[rel] (agents.east) .. controls +(14mm,0) and +(-14mm,12mm) .. node[above]{Infer} (llms.west);
\draw[rel] (agents.east) .. controls +(14mm,0) and +(-14mm,-12mm) .. node[below]{Transcribe} (asr.west);
\draw[dashedrel] (gw) -- node[above]{OIDC} (idp.west);
\draw[dashedrel] (client) |- node[pos=0.25,left]{Corrections} (orch);
\end{tikzpicture}%
}
\caption{C4 container: client, backend, agents, storage, and external ASR/LLM/auth (no datasets/evaluation).}
\label{fig:c4-container}
\end{figure}

\input{images/bpmn}