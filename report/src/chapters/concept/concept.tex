This chapter presents the conceptual foundation of the proposed solution, grounded in the requirements and limitations identified in Chapter 2. The core objective is to transform unstructured narrative input—whether spoken or typed—into structured, schema-conformant templates through a modular, agent-based architecture. Unlike monolithic approaches that attempt end-to-end extraction in a single inference step, the proposed system decomposes the task into specialized subtasks, each handled by a dedicated agent with clearly defined responsibilities and interfaces.

The design is motivated by three critical shortcomings observed in existing work. First, single-model systems lack transparency: when extraction fails, it is difficult to trace whether the error originated in entity recognition, field mapping, or format normalization \cite{du2021template, sun2023slot}. Second, monolithic architectures are brittle under noisy or domain-specific input, as they cannot isolate and recover from localized failures \cite{wang2021spoken}. Third, current solutions provide limited support for iterative refinement, user correction, or learning from domain knowledge, restricting their adaptability in real-world deployments \cite{mialon2023augmented, park2023generative}.

The proposed Invox system addresses these limitations through a five-stage pipeline that enforces separation of concerns while preserving end-to-end coherence. Each agent operates on well-defined inputs and produces structured outputs that serve as inputs to downstream components. This modular design directly supports the six requirements established in Section 2.1: consistency is enforced through dedicated normalization (R1), extraction accuracy benefits from retrieval-augmented context (R2), intermediate artifacts enable traceability (R3), outputs remain editable at the end before final submission (R4), the system learns from historical templates and domain resources (R5), and architectural strategies can be selected to meet latency constraints (R6).

The remainder of this chapter is organized as follows. Section 3.1 derives the conceptual design from the analysis results, explaining how each requirement motivates specific architectural decisions. Section 3.2 describes the five modular agents that comprise the pipeline, detailing their individual responsibilities, inputs, outputs, and internal mechanisms. Section 3.3 presents four architectural strategies—Single-Pass Full Input, Iterative Single-Field Processing, Multi-LLM Consensus (Full), and Multi-LLM Consensus (Iterative)—that instantiate the same pipeline with different trade-offs in accuracy, latency, and cost. Section 3.4 provides the high-level system architecture and design, including context diagrams, container views, and process flows that illustrate how components interact in deployment scenarios. Detailed implementation and experimental validation are deferred to Chapters 4 and 5, respectively.

\section{Concept Derivation from Analysis Results}
\label{sec:concept-derivation}

The conceptual design of the Invox system is derived systematically from the six requirements identified in Section~\ref{sec:requirements} and the gaps observed in existing approaches reviewed in Section~\ref{sec:related-work}. This section traces how each architectural decision directly addresses specific limitations in current template-filling systems, establishing the rationale for a modular, multi-agent design.

\subsection{From Monolithic to Modular Processing}
\label{subsec:monolithic-to-modular}

The analysis in Section~\ref{sec:related-work} revealed that monolithic approaches—where a single LLM performs extraction, normalization, and validation in one inference step—suffer from three fundamental weaknesses. First, they propagate errors across stages without providing mechanisms for localized recovery \cite{sun2023slot}. When entity recognition fails, subsequent field mapping inherits these errors, compounding inaccuracies. Second, they lack transparency: users cannot trace which part of the input led to which output, making error diagnosis difficult and undermining trust \cite{ribeiro2016should}. Third, they provide no intermediate points for user intervention, forcing corrections to be applied post-hoc rather than during processing \cite{amershi2019guidelines}.

These observations directly motivate the decomposition of the template-filling task into discrete, sequential stages. By separating transcription, retrieval, extraction, normalization, and verification into dedicated agents, the system gains three critical capabilities. Errors can be isolated to specific stages, allowing targeted debugging and recovery without reprocessing the entire pipeline. Intermediate outputs become visible and editable, supporting both transparency (R3) and user correction (R4). Finally, individual agents can be improved or replaced independently, enabling iterative refinement without destabilizing the overall architecture.

\subsection{Addressing Consistency Through Dedicated Normalization}
\label{subsec:consistency-normalization}

Requirement R1 demands that heterogeneous inputs—whether informal speech transcripts, chat messages, or multilingual notes—be transformed into uniform, comparable template entries. Existing generative systems often produce stylistically inconsistent outputs because the same model is responsible for both content extraction and format enforcement \cite{huang2024authorship}. This dual responsibility introduces variance: one inference may produce "08/16/2025" while another outputs "August 16, 2025" for the same date, even when guided by identical prompts.

One solution employed in this thesis is to delegate normalization to a specialized Consistency Formatting (CF) agent that operates independently of extraction. Once the Information Extraction (IE) agent proposes candidate values, CF applies deterministic transformations—date standardization, entity canonicalization, and vocabulary alignment—ensuring that all outputs conform to predefined schemas. This separation keeps extraction logic focused on identifying correct content, while formatting logic enforces structural and stylistic uniformity. The result is higher consistency across diverse inputs, directly satisfying R1.

\subsection{Retrieval-Augmented Generation for Domain Adaptation}
\label{subsec:rag-domain-adaptation}

Robust information extraction under noisy, incomplete, and domain-specific conditions \textbf{(R2)} remains a challenge for single-shot LLMs. As discussed in Section~\ref{sec:related-work}, zero-shot models struggle with specialized terminology, abbreviations, and implicit references common in industrial and healthcare environments \cite{wang2021spoken}. Few-shot prompting mitigates some of these issues, but manually curating examples for every deployment scenario is impractical and does not scale across domains \cite{wei2022emergent}.

In Invox, this issue is addressed using a Retrieval-Augmented Generation (RAG) agent positioned between transcription and extraction. The RAG agent queries an indexed corpus of historical templates and domain glossaries, retrieving the $k$ most semantically similar examples for the current input. These examples are then passed to the IE agent as dynamic few-shot context, grounding its reasoning in domain-specific patterns without requiring manual prompt engineering. This design directly supports R2 by improving extraction accuracy on specialized vocabulary, and it also contributes to R5 (learning and adaptation) by leveraging organizational knowledge accumulated over time.

\subsection{Verification for Completeness and Cross-Field Consistency}
\label{subsec:verification-consistency}

Existing template-filling systems rarely validate their own outputs. Once fields are populated, the result is presented to the user without checking for missing required fields, contradictory values, or schema violations \cite{sun2023slot}. This places the entire verification burden on the user, increasing workload and reducing trust.

The Verification (VER) agent addresses this gap by performing three types of checks after normalization. Completeness checks ensure that all required fields have been populated or explicitly marked as unavailable. Cross-field consistency checks detect contradictions, such as an incident date that falls outside the reported shift time. Confidence scoring highlights uncertain extractions, directing user attention to fields most likely to require correction. When issues are detected, VER can trigger a clarification loop, prompting the user for additional input before re-applying CF and VER on the updated values. This design supports R3 (transparency) by making verification criteria explicit, and R4 (user correction) by enabling targeted intervention.

\subsection{Modularity as a Foundation for Flexibility}
\label{subsec:modularity-flexibility}

The analysis in Section~\ref{sec:related-work} showed that no existing system satisfies all six requirements simultaneously. Approaches optimized for accuracy often sacrifice latency \cite{park2023generative}, while methods prioritizing speed tend to compromise on transparency or adaptability \cite{google2024langextract}. These trade-offs are inherent to monolithic designs, where architectural decisions are fixed within a single model and cannot be adjusted post-deployment.

The modular pipeline enables flexible deployment strategies without changing the underlying components. The same five agents can be orchestrated in multiple ways: as a sequential single-pass system for low-latency scenarios, as an iterative slot-parallel architecture for higher accuracy, or as a multi-model consensus mechanism for applications where reliability outweighs cost. The remainder of this section will show how each architectural strategy can be instantiated using the same agent set, and how their respective trade-offs align with different operational requirements. This flexibility directly addresses R6 (usability) by allowing latency and resource constraints to guide strategy selection while preserving modularity for future improvements.

This section summarizes how each architectural decision maps to the requirements established in Chapter~\ref{chap:analysis}. The modular design is not an arbitrary choice, but a direct consequence of the limitations observed in existing work and the operational constraints of real-world deployment contexts. The next section details the five agents that instantiate this design, specifying their inputs, outputs, and internal processing logic.
\section{Four-Agent Architecture}
\label{sec:concept-architecture}

The modular architecture of \textit{Invox} is organised into four agents. Each agent exposes a clear input/output contract, which enables component-level analysis and flexible substitution. This section characterises the roles of the agents and their interconnections, independent of any particular deployment strategy.

\subsection{Speech-to-Text Agent (STT)}
The STT agent converts spoken language into text using a Whisper-based transcription model. It accepts audio recordings as input and outputs transcribed text along with metadata such as confidence scores and timestamps. As the focus of this work is the mapping from unstructured text to template fields, the STT component remains an adapter rather than the core of the architecture. Textual inputs bypass this stage and enter directly into the Information Extraction agent.

\subsection{Information Extraction Agent (IE)}
The IE agent performs the central task of transforming unstructured text into structured template fields. Its input is free-form text, either transcribed or provided directly. Its outputs are candidate slot-value pairs for the target template schema.

The agent identifies relevant entities, relations, and events by prompting a large language model with task-specific instructions. Depending on the strategy (cf.~Section~\ref{sec:concept-strategies}), this can involve one-pass generation of all fields, iterative slot-by-slot processing, or consensus across multiple LLMs. The IE agent is therefore the primary source of architectural variation. 

Design requirements for the IE agent are:
\begin{itemize}
    \item \textbf{Completeness:} all relevant slots must be addressed, even if values are uncertain.
    \item \textbf{Faithfulness:} extracted values must reflect only information present in the source text.
    \item \textbf{Traceability:} intermediate reasoning or supporting spans should be accessible for verification.
\end{itemize}

\subsection{Consistency Formatting Agent (CF)}
The CF agent standardises the candidate values into canonical forms. This includes normalising dates, abbreviations, and entity variants, as well as enforcing schema-level conventions such as country names or role labels. The agent ensures that different surface forms (e.g., ``Sept.~11'' vs. ``11 September 2001'') are mapped to a common representation, which simplifies downstream comparison and evaluation. Although lightweight relative to IE, CF is crucial for robustness when processing large volumes of heterogeneous documents.

\subsection{Verification Agent (VER)}
The VER agent validates the structured output. It accepts the formatted slot-value pairs and produces a verified template plus a set of annotations. Verification comprises:
\begin{itemize}
    \item \textbf{Completeness checks:} identification of missing fields.
    \item \textbf{Consistency checks:} detection of internal contradictions (e.g., differing dates for the same event).
    \item \textbf{Confidence scoring:} estimation of reliability for each field, which can inform whether clarification or user review is required.
\end{itemize}

When low confidence or contradictions are detected, VER triggers a clarification loop. This may involve re-prompting the IE agent with a refined query or requesting additional input from the user. In strategies that employ multiple LLMs, VER also compares alternative outputs and resolves them by applying consensus rules.

\subsection{Agent Interfaces}
Table~\ref{tab:agent-interfaces} summarises the interface of each agent. By separating their responsibilities and outputs, the architecture supports substitution of different models, logging of intermediate states, and detailed error analysis.

\begin{table}[H]
\centering
\begin{tabular}{p{3cm}p{4cm}p{6cm}}
\toprule
\textbf{Agent} & \textbf{Input} & \textbf{Output} \\
\midrule
STT & Audio signal & Transcribed text with metadata \\
IE & Text (transcribed or raw) & Candidate slot-value pairs \\
CF & Candidate slot-value pairs & Normalised slot-value pairs \\
VER & Normalised slot-value pairs & Verified template + confidence + annotations \\
\bottomrule
\end{tabular}
\caption{Agent interfaces in the Invox architecture.}
\label{tab:agent-interfaces}
\end{table}


% Preamble (once, if not already loaded)
% \usepackage{tikz}
% \usetikzlibrary{arrows.meta, positioning}

\begin{figure}[H]
\centering
\resizebox{0.7\linewidth}{!}{%
\begin{tikzpicture}[
    every node/.style={font=\sffamily},
    box/.style={draw, rounded corners=2pt, thick, minimum width=38mm, minimum height=12mm, align=center, fill=blue!7},
    iobox/.style={draw, rounded corners=2pt, thick, minimum width=34mm, minimum height=10mm, align=center, fill=gray!10},
    arrow/.style={-Latex, thick},
    fb/.style={-Latex, thick, dashed},
    node distance=10mm
]
% Inputs (top row)
\node[iobox] (audio) {Audio Input};
\node[iobox, right=25mm of audio] (text) {Text Input};

% Pipeline (vertical stack)
\node[box, below=15mm of audio.center] (stt) {Speech-to-Text (STT)};
\node[box, below=15mm of stt] (ie)  {Information Extraction (IE)};
\node[box, below=15mm of ie]  (cf)  {Consistency Formatting (CF)};
\node[box, below=15mm of cf]  (ver) {Verification (VER)};
\node[iobox, below=15mm of ver] (out) {Structured Template};

% Main straight connections (vertical)
\draw[arrow] (stt) -- (ie);
\draw[arrow] (ie) -- (cf);
\draw[arrow] (cf) -- (ver);
\draw[arrow] (ver) -- (out);

% Input routes (right-angle only)
% Audio -> STT (straight down via left)
\draw[arrow] (audio) |- (stt);
% Text -> IE (right-angle from right input)
\draw[arrow] (text) |- (ie);

% Clarification feedback (from VER back to IE and STT)
% VER -> IE (text clarification)
\draw[fb] (ver.east) -| node[pos=0.25, right]{\scriptsize Clarification (Text)} (ie.east);
% VER -> STT (speech clarification)
\draw[fb] (ver.west) -| node[pos=0.25, left]{\scriptsize Clarification (Speech)} (stt.west);

\end{tikzpicture}%
}
\caption{Vertical agent pipeline with right-angle connectors. The system primarily starts from speech (Audio $\rightarrow$ STT), optionally accepts direct text, and allows verification-driven clarification either as speech (loop to STT) or text (loop to IE).}
\label{fig:agent-pipeline-vertical}
\end{figure}


\section{Architectural Strategies}
\label{sec:architectural-strategies}

While the modular five-agent pipeline defines the general structure of the \textit{Invox} system, its effectiveness depends on how large language models (LLMs) are deployed within the \textit{Information Extraction} and \textit{Verification} stages. Different architectural strategies offer distinct trade-offs in terms of accuracy, robustness, latency, cost, and transparency. This section outlines four strategies and illustrates them using a common example to enable direct comparison.

For all examples, we use the following input derived from the MUC-4 benchmark:  
\textit{``On March 3, 1992, in Bogotá, a powerful car bomb exploded outside the Ministry of Defense, damaging nearby buildings and injuring 25 people, though no fatalities were reported. Authorities suspect a left-wing guerrilla group, but responsibility remains unconfirmed.''}

This example introduces multiple attributes (date, location, event type, casualties, suspected perpetrators, and epistemic uncertainty), allowing us to highlight the strengths and weaknesses of each strategy under realistic conditions.

\subsection{Strategy S1: Single-Pass (Full-Input, Single-LLM)}
\label{subsec:strategy-s1}

In the simplest approach, a single LLM receives the complete transcript—along with few-shot examples retrieved by the RAG agent—and directly generates all template fields in one inference pass. The extracted values then proceed through Consistency Formatting (CF) and Verification (VER) before finalization.

\begin{figure}[H]
  \centering
  \includegraphics[width=1.0\linewidth]{images/single-llm-full-input.drawio.pdf}
  \caption{Strategy: Full-Input, Single-LLM}
  \label{fig:single-llm-full-input}
\end{figure}

\textbf{Advantages.}  
This strategy is computationally efficient, requiring only one IE model call. It minimizes latency and API costs, making it suitable for high-throughput scenarios. The inclusion of RAG-retrieved examples improves domain adaptation compared to zero-shot prompting, while CF and VER stages provide downstream quality assurance.

\textbf{Limitations.}  
If the LLM misinterprets a critical ambiguity—such as treating ``suspected guerrilla group'' as a confirmed perpetrator—the error propagates through CF (which normalizes the incorrect value) and may only be flagged by VER if confidence thresholds are exceeded. The single inference point offers no redundancy, making the system vulnerable to model-specific biases or hallucinations \cite{du2020event}.

\subsection{Strategy S2: Iterative (Slot-wise, Single-LLM)}
\label{subsec:strategy-s2}

Each template slot is extracted by an independent LLM call with a slot-specific prompt. Slots can be processed sequentially or in parallel. The RAG agent retrieves relevant examples once at the beginning, and these are included in every slot-level prompt. Results are then passed to CF and VER for normalization and validation.

\begin{figure}[H]
  \centering
  \includegraphics[width=1.0\linewidth]{images/single-llm-slot-wise.drawio.pdf}
  \caption{Strategy: Slot-wise, Single-LLM}
  \label{fig:single-llm-slot-wise}
\end{figure}

\textbf{Advantages.}  
Slot independence isolates errors: if the perpetrator field is uncertain, only that slot requires re-extraction. This improves transparency (R3) by making it clear which fields are problematic. Parallelization across slots can reduce wall-clock latency on multi-core systems, though total computational cost increases. The strategy also supports slot-specific prompt engineering, allowing prompts to be tuned for date extraction, entity recognition, or casualty parsing independently.

\textbf{Limitations.}  
The approach increases inference cost linearly with the number of slots. For templates with 15–20 fields, this becomes expensive. Additionally, slot-level prompts lack cross-field context: the model extracting casualties does not see the perpetrator information, which may reduce coherence in cases where fields are interdependent \cite{sun2023slot}.

\subsection{Strategy S3: Multi-LLM Consensus (Full-Input)}
\label{subsec:strategy-s3}


\begin{figure}[H]
  \centering
  \includegraphics[width=1.0\linewidth]{images/multiple-llm-full-input.drawio.pdf}
  \caption{Strategy: Full-Input, Multiple-LLM}
  \label{fig:multiple-llm-full-input}
\end{figure}

Multiple LLMs—either different models (e.g., GPT-4, Claude, DeepSeek) or multiple runs of the same model with varied prompts—each generate a complete template from the full transcript and RAG-retrieved examples. Their outputs are compared by the Internal Verifier which has results from multiple llms, text input and the RAG examples, which selects the most consistent or reliable values using consensus rules such as majority voting or confidence-weighted aggregation.

\textbf{Advantages.}  
Ensemble diversity reduces systematic bias: if one model hallucinates or misinterprets ambiguous phrasing, others may produce more faithful outputs. Consensus mechanisms increase robustness, particularly when models disagree on uncertain fields. This strategy also preserves cross-field context, as each model sees the full input during extraction \cite{wu2023autoagents, park2023generative}.

\textbf{Limitations.}  
Computational cost increases linearly with the number of models. If all models converge on the same misinterpretation—due to shared training biases or ambiguous input phrasing—consensus provides no advantage. The strategy also requires careful tuning of aggregation rules: simple majority voting may discard nuanced or hedged outputs (e.g., "suspected") in favor of overly confident but incorrect answers.

\subsection{Strategy S4: Multi-LLM Consensus (Slot-wise)}
\label{subsec:strategy-s4}

Each slot is extracted independently by multiple LLMs, and consensus is reached per field before aggregating results. This combines the granularity of S2 (slot-wise processing) with the robustness of S3 (multi-model consensus). Their outputs are compared by the Internal Verifier which has results from multiple llms, text input and the RAG examples, which selects the most consistent or reliable values using consensus rules such as majority voting or confidence-weighted aggregation. Results are then passed through CF and VER. 

\textbf{Advantages.}  
This strategy offers the highest reliability by applying ensemble methods at the most granular level. Slot-level consensus allows diverse models to specialize: one model may excel at date parsing, another at casualty extraction. Error isolation remains strong, as each field is independently verified. The approach is well-suited for safety-critical domains where accuracy justifies cost.

\textbf{Limitations.}  
Computational cost scales as (number of slots) × (number of models), making this the most expensive strategy. For a 15-field template with 3 models, this requires 45 LLM calls per document. Latency increases unless aggressive parallelization is employed. Additionally, consensus logic becomes more complex when models produce semantically equivalent but syntactically different outputs (e.g., "25 injured" vs. "25 people hurt"), requiring robust normalization before voting.

\begin{figure}[H]
  \centering
  \includegraphics[width=1.0\linewidth]{images/multiple-llm-slot-wise.drawio.pdf}
  \caption{Strategy: Slot-wise, Multiple-LLM}
  \label{fig:multiple-llm-slot-wise}
\end{figure}

\subsection{Summary and Trade-Off Analysis}
\label{subsec:strategy-summary}

\begin{table}[H]
\centering
\renewcommand{\arraystretch}{1.3}
\setlength{\tabcolsep}{10pt}
\small
\begin{tabular}{|
>{\centering\arraybackslash}m{2.6cm}|
>{\centering\arraybackslash}m{1.6cm}|
>{\centering\arraybackslash}m{1.8cm}|
>{\centering\arraybackslash}m{1.6cm}|
>{\centering\arraybackslash}m{1.6cm}|
>{\centering\arraybackslash}m{1.6cm}|
}
\hline
\textbf{Strategy} & \textbf{LLM Calls} & \textbf{Error Isolation} &
\textbf{Robustness} & \textbf{Latency} & \textbf{Cost} \\
\hline

S1: Single-Pass & 1 & Low & Low & Low & Low \\
\hline

S2: Iterative & $n$ (slots) & High & Medium & Medium & Medium \\
\hline

S3: Consensus (Full) & $m$ (models) & Low & High & Medium & High \\
\hline

S4: Consensus (Slot-Level) & $n \times m$ & High & High & High & High \\
\hline

\end{tabular}
\caption{Comparison of architectural strategies, showing how different orchestration approaches trade off robustness, latency, and cost depending on the number of LLM calls and the level of error isolation.}
\label{tab:architectural-strategy-comparison}
\end{table}


Deployment requirements determine which strategy is most appropriate. \textbf{S1} is ideal for high-throughput scenarios where speed and cost outweigh accuracy needs. \textbf{S2} offers strong error isolation and fine-grained debugging, making it suitable when specific fields frequently cause failures. \textbf{S3} provides a balanced compromise between robustness and efficiency, supporting general-purpose use cases. \textbf{S4} delivers the highest reliability by combining per-slot consensus with multiple models, making it the preferred choice for safety-critical or compliance-sensitive workflows (e.g., medical records or regulatory reporting).


Chapter~5 empirically evaluates these strategies on the MUC-4 benchmark and quantifies their performance across the six requirements (R1--R6) established in Chapter~2.

\section{System Architecture and Design}
\label{sec:concept-design}

This section presents the high-level architecture of the Invox system using the C4 model, which structures architectural documentation into hierarchical levels: context (system boundaries and external actors), containers (major components and their interactions), and process (dynamic execution flow). These views provide a comprehensive understanding of how the conceptual design translates into a deployable architecture.

\subsection{Context View: System Boundaries and External Actors}
\label{subsec:context-view}

\begin{figure}[H]
  \centering
  \includegraphics[width=1\linewidth]{images/c4_context.drawio.pdf}
  \caption{Context diagram showing system boundaries and external actors.}
  \label{fig:c4-context}
\end{figure}

The context diagram (Figure~\ref{fig:c4-context}) positions Invox within its operational environment. The system interacts with three categories of external actors:

\textbf{Primary users} are domain experts—such as factory operators, medical staff, or administrative personnel—who provide unstructured input (speech or text) and review populated templates. They interact with the system through a web-based interface or mobile application.

\textbf{Data sources} include historical template repositories, domain-specific glossaries, and organizational knowledge bases. These are accessed by the RAG agent to retrieve contextually relevant examples. In regulated environments, these repositories may reside behind secure access controls.

\textbf{External services} consist of third-party APIs for LLM inference (OpenAI GPT-4, Anthropic Claude, DeepSeek), speech recognition (OpenAI Whisper), and search infrastructure (OpenSearch). The system communicates with these services over REST APIs.

\subsection{Container View: Internal Component Structure}
\label{subsec:container-view}

The container diagram (Figure~\ref{fig:c4-container}) decomposes Invox into its major internal components and clarifies how requests flow through the system.

\textbf{Client access.} Users interact with the system via the \emph{API Gateway}, which supports HTTP, HTTPS, or tRPC protocols. This is the only entry point for structured input requests. For raw audio, users may also call the \emph{STT Service} directly to obtain transcribed text.

\textbf{STT Service.} The STT service wraps the Whisper model, invokes the external ASR provider, and returns transcribed text enriched with metadata such as confidence scores, timestamps, and language detection. Transcribed text is returned to the client for display or routed back into the pipeline.

\textbf{RAG Service.} Once the API Gateway accepts a request, it forwards it into the backend pipeline starting with the RAG service. This service embeds the input, queries the \emph{Vector Index}, and enriches the request with semantically relevant examples.

\textbf{IE Service.} Using the enriched context, the IE service constructs prompts and invokes the configured LLM providers. Depending on the strategy (S1–S4), this may involve a single call, slot-wise calls, or multiple model invocations. Results are returned as candidate slot-value pairs.

\textbf{CF Service.} The CF service normalizes the extracted values (e.g., dates, entities) and enforces schema constraints to produce a consistent intermediate template.

\textbf{VER Service.} 
The verification service ensures completeness, consistency, and confidence of the template. It assigns confidence scores, detects conflicts, flags missing fields, and can trigger clarification requests via the API Gateway. Finalized templates, enriched with quality metadata, are stored in both the \emph{Template Repository} and the \emph{Vector Index}. This service is internal-only and not directly accessible to clients.

\textbf{Data stores.} The system relies on two primary data stores. A \emph{Template Repository} maintains finalized templates that can be retrieved and reused in future interactions, while a \emph{Vector Index} stores dense embeddings to support retrieval-augmented generation and similarity-based lookup of relevant past cases.

\textbf{End-to-end flow.} After verification, the finalized template is stored, indexed, and routed back through the API Gateway to the client as the official response.

\begin{sidewaysfigure}
  \centering
  \includegraphics[width=1\linewidth]{images/c4_container.drawio.pdf}
  \caption{container diagram showing request flow through internal components and external dependencies.}
  \label{fig:c4-container}
\end{sidewaysfigure}

\subsection{Process View: Workflow and Decision Logic}
\label{subsec:process-view}

The process diagram (Figure~\ref{fig:bpmn}) presents the dynamic execution flow. The workflow follows this structure:

\textbf{Input reception.} User submits audio or text. Audio is processed by the STT service; text requests proceed via the API Gateway into the backend pipeline.

\textbf{Retrieval.} The RAG service queries the vector index and returns top-$k$ examples. If retrieval fails, the system falls back to zero-shot prompting.

\textbf{Extraction.} The IE service invokes the LLM(s) according to the chosen strategy. For multi-model strategies, all responses are collected before proceeding.

\textbf{Formatting.} CF normalizes extracted values and enforces schema constraints, flagging any that cannot be normalized for manual review.

\textbf{Verification.} VER checks consistency, completeness, and confidence. Possible outcomes are: (1) Pass—template is complete and returned; (2) Clarification required—system prompts user and re-executes IE $\rightarrow$ CF $\rightarrow$ VER; (3) Low confidence—template returned with uncertain fields for user review.

\textbf{Finalization.} Verified templates are stored in the repository, indexed for future retrieval, and returned to the client. Users may still edit fields before final submission.

\begin{sidewaysfigure}
  \centering
  \includegraphics[width=\linewidth]{images/bpmn_process_flow.pdf}
  \caption{BPMN diagram showing workflow and decision logic.}
  \label{fig:bpmn}
\end{sidewaysfigure}
