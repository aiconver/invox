\section{System Architecture and Design}
\label{sec:concept-design}

This section presents the high-level architecture of the Invox system using the C4 model, which structures architectural documentation into hierarchical levels: context (system boundaries and external actors), containers (major components and their interactions), and process (dynamic execution flow). These views provide a comprehensive understanding of how the conceptual design translates into a deployable architecture.

\subsection{Context View: System Boundaries and External Actors}
\label{subsec:context-view}

The context diagram (Figure~\ref{fig:c4-context}) positions Invox within its operational environment. The system interacts with three categories of external actors:

\textbf{Primary users} are domain experts—such as factory operators, medical staff, or administrative personnel—who provide unstructured input (speech or text) and review populated templates. They interact with the system through a web-based interface or mobile application.

\textbf{Data sources} include historical template repositories, domain-specific glossaries, and organizational knowledge bases. These are accessed by the RAG agent to retrieve contextually relevant examples. In regulated environments, these repositories may reside behind secure access controls.

\textbf{External services} consist of third-party APIs for LLM inference (OpenAI GPT-4, Anthropic Claude, DeepSeek), speech recognition (OpenAI Whisper), and search infrastructure (OpenSearch). The system communicates with these services over REST APIs.

\begin{figure}[H]
  \centering
  \includegraphics[width=1\linewidth]{images/c4_context.drawio.pdf}
  \caption{Context diagram illustrating the system boundary of Invox and its interactions with external actors such as users, authentication services, LLM providers, speech-to-text components, and organizational knowledge sources, highlighting all inputs the system depends on.}
  \label{fig:c4-context}
\end{figure}

\subsection{Container View: Internal Component Structure}
\label{subsec:container-view}

The container diagram (Figure~\ref{fig:c4-container}) outlines Invox’s main components and the flow of requests through the system.

\textbf{Client access.} Users interact with Invox via the \emph{API Gateway}, the entry point for all structured requests over HTTP/HTTPS/tRPC. Audio may also be sent directly to the \emph{STT Service} for transcription.

\textbf{STT Service.} Speech-to-text processing is handled by a Whisper-based component that wraps the external ASR provider and returns transcriptions with metadata such as confidence scores and timestamps. Output is either returned to the client or forwarded into the pipeline.

\textbf{RAG Service.} Once a request enters the backend, the first step is retrieval. The input is embedded and matched against the \emph{Vector Index} to obtain semantically relevant examples that support downstream extraction.

\textbf{IE Service.} Using this enriched context, the IE component builds prompts and invokes the configured LLM providers. Depending on the selected strategy (S1–S4), it may perform a single call, slot-wise calls, or multi-model consensus to produce candidate slot–value pairs.

\textbf{CF Service.} Extracted values are then normalized and validated. This stage standardizes formats (e.g., dates, entities) and enforces schema constraints to produce a consistent intermediate template.

\textbf{VER Service.} Verification is performed as a dedicated internal step. It assigns confidence scores, checks completeness and consistency, flags problematic fields, and triggers clarification prompts when needed. Verified templates, along with metadata, are stored in both the \emph{Template Repository} and the \emph{Vector Index}.

\textbf{Data stores.} Invox uses two storage layers: a \emph{Template Repository} for finalized templates and a \emph{Vector Index} for dense embeddings used during retrieval.

\textbf{End-to-end flow.} After verification, the template is persisted, indexed, and returned to the client via the API Gateway. Manual edits remain possible before final submission.

\begin{sidewaysfigure}
  \centering
  \includegraphics[width=1\linewidth]{images/c4_container.drawio.pdf}
  \caption{Container diagram illustrating how requests move through Invox’s internal services—STT, RAG, IE, CF, and VER—and how these components interact with external dependencies such as LLM providers, storage layers, and the client-facing API Gateway.}
  \label{fig:c4-container}
\end{sidewaysfigure}

\subsection{Process View: Workflow and Decision Logic}
\label{subsec:process-view}

The process diagram (Figure~\ref{fig:bpmn}) illustrates the dynamic execution flow that connects all stages of the Invox pipeline.

\textbf{Input reception.} The workflow starts when users submit either audio or text. Audio is routed through the STT service for transcription, while text requests enter through the API Gateway and continue directly into the backend pipeline.

\textbf{Retrieval.} After the initial preprocessing, the RAG service performs a similarity search over the vector index to obtain the top-$k$ relevant examples. If no suitable matches are found, the system automatically reverts to a zero-shot prompting.

\textbf{Extraction.} With the retrieved examples available, the IE service invokes one or more LLM calls according to the selected strategy. For multi-model or consensus configurations, all model responses are gathered before the pipeline proceeds.

\textbf{Formatting.} The next stage involves normalization and schema enforcement. CF standardizes extracted values, validates them against the schema, and flags any elements that cannot be normalized for later user review.

\textbf{Verification.} VER checks the template for consistency, completeness, and confidence. Outcomes include: (1) a verified template returned directly to the user, (2) a clarification loop that requests missing information and re-runs IE $\rightarrow$ CF $\rightarrow$ VER, or (3) a low-confidence output with uncertain fields highlighted for review.

\textbf{Finalization.} Once verification is complete, the resulting template is stored in the repository, indexed for future retrieval, and returned to the client. Users may still make manual adjustments before final submission.

\begin{sidewaysfigure}
  \centering
  \includegraphics[width=\linewidth]{images/bpmn_process_flow.pdf}
  \caption{BPMN diagram illustrating the end-to-end workflow of Invox, including input handling, retrieval, extraction, formatting, verification, and the decision logic that governs clarification loops and final template delivery.}
  \label{fig:bpmn}
\end{sidewaysfigure}
